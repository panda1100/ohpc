With the \OHPC{} repository enabled, we can now begin adding desired components onto the
{\em master} server. This repository provides a number of aliases that group
logical components together in order to help aid in this process. For
reference, a complete list of available group aliases and RPM packages available
via \OHPC{} are provided in Appendix~\ref{appendix:manifest}. To add
support for provisioning services, the following commands illustrate addition
of a common base package followed by the Warewulf provisioning system.

%\nottoggle{isCentOS}{\clearpage}

% begin_ohpc_run
% ohpc_comment_header Add baseline OpenHPC and provisioning services \ref{sec:add_provisioning}
\iftoggleverb{isWarewulf4}
% Warewulf v4
\begin{lstlisting}[language=bash,keywords={}]
    # Install base meta-packages
    [sms](*\#*) (*\install*) ohpc-base
    [sms](*\#*) (*\install*) warewulf-ohpc
    \end{lstlisting}
\else
% Warewulf v3
\begin{lstlisting}[language=bash,keywords={}]
# Install base meta-packages
[sms](*\#*) (*\install*) ohpc-base
[sms](*\#*) (*\install*) ohpc-warewulf
\end{lstlisting}
\fi
% end_ohpc_run


