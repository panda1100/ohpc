\documentclass[letterpaper]{article}
\usepackage{common/ohpc-doc}
\setcounter{secnumdepth}{5}
\setcounter{tocdepth}{5}

% Include git variables
\input{vc.tex}

% Define Base OS and other local macros
\newcommand{\baseOS}{Rocky 9.2}
\newcommand{\OSRepo}{Rocky\_9.2}
\newcommand{\OSTree}{EL\_9}
\newcommand{\OSTag}{el9}
\newcommand{\baseos}{rocky9.2}
\newcommand{\baseosshort}{rocky9}
\newcommand{\provisioner}{Warewulf}
\newcommand{\provheader}{\provisioner{}}
\newcommand{\rms}{SLURM}
\newcommand{\rmsshort}{slurm}
\newcommand{\arch}{x86\_64}

% Define package manager commands
\newcommand{\pkgmgr}{yum}
\newcommand{\addrepo}{wget -P /etc/yum.repos.d}
\newcommand{\chrootaddrepo}{wget -P \$CHROOT/etc/yum.repos.d}
\newcommand{\clean}{yum clean expire-cache}
\newcommand{\chrootclean}{yum --installroot=\$CHROOT clean expire-cache}
\newcommand{\install}{yum -y install}
\newcommand{\chrootinstall}{yum -y --installroot=\$CHROOT install}
\newcommand{\groupinstall}{yum -y groupinstall}
\newcommand{\groupchrootinstall}{yum -y --installroot=\$CHROOT groupinstall}
\newcommand{\remove}{yum -y remove}
\newcommand{\upgrade}{yum -y upgrade}
\newcommand{\chrootupgrade}{yum -y --installroot=\$CHROOT upgrade}
\newcommand{\tftppkg}{syslinux-tftpboot}
\newcommand{\beegfsrepo}{https://www.beegfs.io/release/beegfs\_7.2.1/dists/beegfs-rhel8.repo}

% boolean for os-specific formatting
\toggletrue{isCentOS}
\toggletrue{isCentOS_ww_slurm_x86}
\toggletrue{isSLURM}
\toggletrue{isWarewulf}
\toggletrue{isWarewulf4}
\toggletrue{isx86}
\toggletrue{isCentOS_x86}

\begin{document}
\graphicspath{{common/figures/}}
\thispagestyle{empty}

% Title Page
\input{common/title}
% Disclaimer 
\input{common/legal} 

\newpage
\tableofcontents
\newpage

% Introduction  --------------------------------------------------

\section{Introduction} \label{sec:introduction}
\input{common/install_header}
\input{common/intro} \\

\input{common/base_edition/edition}
\input{common/audience}
\input{common/requirements}
\input{common/inputs}

% begin_ohpc_run
% ohpc_validation_newline
% ohpc_validation_comment Verify OpenHPC repository has been enabled before proceeding
% ohpc_validation_newline
% ohpc_command yum repolist | grep -q OpenHPC
% ohpc_command if [ $? -ne 0 ];then
% ohpc_command    echo "Error: OpenHPC repository must be enabled locally"
% ohpc_command    exit 1
% ohpc_command fi
% end_ohpc_run

% Base Operating System --------------------------------------------

\section{Install Base Operating System (BOS)}
\input{common/bos}

%\clearpage 
% begin_ohpc_run
% ohpc_validation_newline
% ohpc_validation_comment Disable firewall 
\begin{lstlisting}[language=bash,keywords={}]
[sms](*\#*) systemctl disable firewalld
[sms](*\#*) systemctl stop firewalld
\end{lstlisting}
% end_ohpc_run

% ------------------------------------------------------------------

\section{Install \OHPC{} Components} \label{sec:basic_install}
\input{common/install_ohpc_components_intro.tex}

\subsection{Enable \OHPC{} repository for local use} \label{sec:enable_repo}
\input{common/enable_ohpc_repo}
In addition to the \OHPC{}
\iftoggle{isxCAT}{and \xCAT{} package repositories,}{package repository,}
the {\em master} host also requires access to the standard base OS distro
repositories in order to resolve necessary dependencies. For \baseOS{}, the
requirements are to have access to the BaseOS, Appstream, Extras, CRB,
and EPEL repositories for which mirrors are freely available online:

\begin{itemize*}
\item Rocky-9
  (e.g. \href{http://download.rockylinux.org/pub/rocky/9/}
             {\color{blue}{http://download.rockylinux.org/pub/rocky/9/}} )
\item EPEL 9 (e.g. \href{http://download.fedoraproject.org/pub/epel/9/}
                        {\color{blue}{http://download.fedoraproject.org/pub/epel/9/}} )
\end{itemize*}

\noindent The public EPEL repository will be enabled automatically upon
installation of the \texttt{ohpc-release} package. Note that this does depend
on the Rocky Extras repository, which is shipped with Rocky and is typically
enabled by default.  In contrast, the CRB repository is typically
disabled in a standard install, but can be enabled from EPEL as follows:

\begin{lstlisting}[language=bash,literate={-}{-}1,keywords={},upquote=true]
[sms](*\#*) crb enable
\end{lstlisting}

\input{common/automation}


\subsection{Add provisioning services on {\em master} node} \label{sec:add_provisioning}
With the \OHPC{} repository enabled, we can now begin adding desired components onto the
{\em master} server. This repository provides a number of aliases that group
logical components together in order to help aid in this process. For
reference, a complete list of available group aliases and RPM packages available
via \OHPC{} are provided in Appendix~\ref{appendix:manifest}. To add
support for provisioning services, the following commands illustrate addition
of a common base package followed by the Warewulf provisioning system.

%\nottoggle{isCentOS}{\clearpage}

% begin_ohpc_run
% ohpc_comment_header Add baseline OpenHPC and provisioning services \ref{sec:add_provisioning}
\iftoggleverb{isWarewulf4}
% Warewulf v4
\begin{lstlisting}[language=bash,keywords={}]
    # Install base meta-packages
    [sms](*\#*) (*\install*) ohpc-base
    [sms](*\#*) (*\install*) warewulf-ohpc
    \end{lstlisting}
\else
% Warewulf v3
\begin{lstlisting}[language=bash,keywords={}]
# Install base meta-packages
[sms](*\#*) (*\install*) ohpc-base
[sms](*\#*) (*\install*) ohpc-warewulf
\end{lstlisting}
\fi
% end_ohpc_run



\input{common/enable_pxe}
\input{common/time}

\vspace*{0.15cm}
\subsection{Add resource management services on {\em master} node} \label{sec:add_rm}
\input{common/install_slurm}

\subsection{Optionally add \InfiniBand{} support services on {\em master} node} \label{sec:add_ofed}
\input{common/ibsupport_sms_centos}

\subsection{Optionally add \OmniPath{} support services on {\em master} node} \label{sec:add_opa}
\input{common/opasupport_sms_centos}

\vspace*{-0.15cm}
\subsection{Complete basic Warewulf setup for {\em master} node} \label{sec:setup_ww}
At this point, all of the packages necessary to use \Warewulf{} on the {\em
master} host should be installed. Next, we need to update several
configuration files in order to allow \Warewulf{} to work with \baseOS{} and to
support local provisioning using a second private interface (refer to
Figure~\ref{fig:physical_arch}).
%\vspace*{-0.05cm}
\iftoggleverb{isWarewulf}
% Warewulf v3
\begin{center}
\begin{tcolorbox}[]
\small
By default, \Warewulf{} is configured to
provision over the \texttt{eth1} interface and the steps below include updating
this setting to override with a potentially alternatively-named interface specified by
\texttt{\$\{sms\_eth\_internal\}}.
\end{tcolorbox}
\end{center}
\else
% Warewulf v4
\fi

% begin_ohpc_run
% ohpc_comment_header Complete basic Warewulf setup for master node \ref{sec:setup_ww}
%\begin{verbatim}

\iftoggleverb{isWarewulf4}
% Warewulf v4
\begin{lstlisting}[language=bash,literate={-}{-}1,keywords={},upquote=true,keepspaces]
# Replace “ipaddr: 192.168.200.1” with the IP address of the cluster interface
[sms](*\#*) perl -pi -e "s/ipaddr: 192.168.200.1/ibaddr: ${sms_ip}/" /etc/warewulf/warewulf.conf

# Update the netmask field, if needed
[sms](*\#*) perl -pi -e "s/netmask: 255.255.255.0/netmask: ${sms_netmask}/" /etc/warewulf/warewulf.conf

# Update the network field
[sms](*\#*) perl -pi -e "s/network: 192.168.200.0/network: ${sms_network}/" /etc/warewulf/warewulf.conf

# Update the DHCP ranges, if needed
[sms](*\#*) perl -pi -e "s/range start: 192.168.200.50/range start: ${dhcp_range_start}/" /etc/warewulf/warewulf.conf
[sms](*\#*) perl -pi -e "s/range end: 192.168.200.99/range end: ${dhcp_range_end}/" /etc/warewulf/warewulf.conf

# Change "mount" option for "/opt" to "true"
[sms](*\#*) perl -pi -e "s/mount: false/mount: true/" /etc/warewulf/warewulf.conf

# Enable and start Warewulf service
[sms](*\#*) systemctl enable warewulfd --now

# Implement all configuration changes
[sms](*\#*) wwctl configure --all
\end{lstlisting}
\else
% Warewulf v3
\begin{lstlisting}[language=bash,literate={-}{-}1,keywords={},upquote=true,keepspaces]
# Configure Warewulf provisioning to use desired internal interface
[sms](*\#*) perl -pi -e "s/device = eth1/device = ${sms_eth_internal}/" /etc/warewulf/provision.conf

# Enable internal interface for provisioning
[sms](*\#*) ip link set dev ${sms_eth_internal} up
[sms](*\#*) ip address add ${sms_ip}/${internal_netmask} broadcast + dev ${sms_eth_internal}

# Restart/enable relevant services to support provisioning
[sms](*\#*) systemctl enable httpd.service
[sms](*\#*) systemctl restart httpd
[sms](*\#*) systemctl enable dhcpd.service
[sms](*\#*) systemctl enable tftp.socket
[sms](*\#*) systemctl start tftp.socket
\end{lstlisting}
\fi
%\end{verbatim}
% end_ohpc_run



\subsection{Define {\em compute} image for provisioning}
\input{common/warewulf_mkchroot_rocky}

\subsubsection{Add \OHPC{} components} \label{sec:add_components}
\input{common/add_to_compute_chroot_intro}

%\newpage
% begin_ohpc_run
% ohpc_validation_comment Add SLURM and other components to compute instance
\begin{lstlisting}[language=bash,literate={-}{-}1,keywords={},upquote=true]
# copy credential files into $CHROOT to ensure consistent uid/gids for slurm/munge at
# install. Note that these will be synchronized with future updates via the provisioning system.
[sms](*\#*) cp /etc/passwd /etc/group  $CHROOT/etc

# Add Slurm client support meta-package and enable munge
[sms](*\#*) (*\chrootinstall*) ohpc-slurm-client
[sms](*\#*) chroot $CHROOT systemctl enable munge

# Register Slurm server with computes (using "configless" option)
[sms](*\#*) echo SLURMD_OPTIONS="--conf-server ${sms_ip}" > $CHROOT/etc/sysconfig/slurmd

# Add Network Time Protocol (NTP) support
[sms](*\#*) (*\chrootinstall*) chrony
# Identify master host as local NTP server
[sms](*\#*) echo "server ${sms_ip} iburst" >> $CHROOT/etc/chrony.conf

# Add kernel drivers (matching kernel version on SMS node)
[sms](*\#*) (*\chrootinstall*) kernel-`uname -r`

# Include modules user environment
[sms](*\#*) (*\chrootinstall*) lmod-ohpc
\end{lstlisting}
% end_ohpc_run

\vspace*{.2cm}
\subsubsection{Customize system configuration} \label{sec:master_customization}
\input{common/warewulf_chroot_customize_centos}
\input{common/oneapi_mountpoint}
\input{common/restart_nfs}

% Additional commands when additional computes are requested

% begin_ohpc_run
% ohpc_validation_newline
% ohpc_validation_comment Update basic slurm configuration if additional computes defined
% ohpc_command if [ ${num_computes} -gt 4 ];then
% ohpc_command    perl -pi -e "s/^NodeName=(\S+)/NodeName=${compute_prefix}[1-${num_computes}]/" /etc/slurm/slurm.conf
% ohpc_command    perl -pi -e "s/^PartitionName=normal Nodes=(\S+)/PartitionName=normal Nodes=${compute_prefix}[1-${num_computes}]/" /etc/slurm/slurm.conf

% ohpc_command fi
% end_ohpc_run

%\clearpage
\subsubsection{Additional Customization ({\em optional})} \label{sec:addl_customizations}
\input{common/compute_customizations_intro}

%\clearpage
\paragraph{Enable \InfiniBand{} drivers}
\input{common/ibsupport_compute_centos.tex}

\paragraph{Enable \OmniPath{} drivers}
\input{common/opasupport_compute_centos.tex}

\vspace*{0.28cm}
\paragraph{Increase locked memory limits}
\input{common/memlimits}

\vspace*{-.17cm}
\paragraph{Enable ssh control via resource manager} 
\input{common/slurm_pam}

\vspace*{-.17cm}
\paragraph{Add \beegfs{}} \label{sec:add_beegfs}
\input{common/install_beegfs_client_centos}

\paragraph{Add \Lustre{} client} \label{sec:lustre_client}
\input{common/lustre-client}
%\vspace*{0.25cm}
\input{common/lustre-client-centos}
\input{common/lustre-client-post}

%\vspace*{.45cm}
\paragraph{Enable forwarding of system logs} \label{sec:add_syslog}
\input{common/syslog}

\paragraph{Add \Nagios{} monitoring} \label{sec:add_nagios}
\input{common/nagios}

\paragraph{Add \clustershell{}}
\input{common/clustershell}

\paragraph{Add \genders{}}
\input{common/genders}

\paragraph{Add Magpie}
\input{common/magpie}

\paragraph{Add \conman{}} \label{sec:add_conman}
\input{common/conman}

\paragraph{Add \nhc{}} \label{sec:add_nhc}
\input{common/nhc}
\input{common/nhc_slurm}

\vspace*{0.3cm}
\paragraph{Add \GEOPM{}} \label{sec:add_geopm}
\input{common/geopm_config}

%\clearpage
\subsubsection{Import files} \label{sec:file_import}
\input{common/import_ww_files}
%\vspace*{0.3cm}
\input{common/import_ww_files_slurm}
\input{common/import_ww_files_ib_centos}
%\vspace*{0.3cm}
\input{common/finalize_provisioning}
%\vspace*{0.2cm}
\input{common/add_ww_hosts_intro}
\input{common/add_ww_hosts_slurm}
\input{common/add_ww_hosts_finalize}

%\clearpage
\subsubsection{Optional kernel arguments} \label{sec:optional_kargs}
\input{common/charliecloud_centos_warewulf_post}
\input{common/conman_post}
\input{common/kargs_post}

\vspace*{-0.1cm}
\subsubsection{Optionally configure stateful provisioning}
\input{common/stateful}

%\vspace*{-0.4cm}
\clearpage
\subsection{Boot compute nodes} \label{sec:boot_computes}
\input{common/reset_computes} 

%\clearpage
\section{Install \OHPC{} Development Components}
\input{common/dev_intro.tex}

%\vspace*{-0.15cm}
\subsection{Development Tools} \label{sec:install_dev_tools}
\input{common/dev_tools}

\vspace*{-0.15cm}
\subsection{Compilers} \label{sec:install_compilers}
\input{common/compilers}

%\clearpage
\subsection{MPI Stacks} \label{sec:mpi}
\input{common/mpi_slurm}

\subsection{Performance Tools} \label{sec:install_perf_tools}
\input{common/perf_tools_with_geopm}

\subsection{Setup default development environment}
\input{common/default_dev}

\vspace*{0.3cm}
\subsection{3rd Party Libraries and Tools} \label{sec:3rdparty}
\input{common/third_party_libs_intro}
\input{common/third_party_libs_petsc_centos}
\input{common/third_party_libs}
\vspace*{0.1cm}
\input{common/third_party_mpi_libs_x86}
\vspace*{0.5cm}
\subsection{Optional Development Tool Builds} \label{sec:3rdparty_intel}
\input{common/oneapi_enabled_builds.tex}

\clearpage
\section{Resource Manager Startup} \label{sec:rms_startup}
\input{common/slurm_startup}

\section{Post-boot compute node configuration} \label{sec:post_boot}
\input{common/post_boot}

\section{Run a Test Job} \label{sec:test_job}
\input{common/slurm_test_job}

\clearpage
\appendix
%\section*{Appendices}
{\bf \LARGE \centerline{Appendices}} \vspace*{0.2cm}

\addcontentsline{toc}{section}{Appendices}
\renewcommand{\thesubsection}{\Alph{subsection}}

\input{common/automation_appendix}
\input{common/upgrade}
\input{common/test_suite}
\input{common/customization_appendix_centos}
\input{manifest}
\input{common/signature}


\end{document}

